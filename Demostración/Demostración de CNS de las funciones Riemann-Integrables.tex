\documentclass[10pt]{article}

\title{Demostración de la condición necesaria y suficiente de las funciones Riemann-Integrables}
\date{2025-03-17}
\author{Adrián Estevez Álvarez; Javier Fontes Basabe  }

\usepackage[spanish]{babel}
\usepackage[utf8]{inputenc}
\usepackage{amsmath, amssymb}
\usepackage[margin = 3cm]{geometry}
\begin{document}
\begin{titlepage}
		\centering
			{\bfseries\LARGE Universidad de La Habana \par}
				\vspace{1cm}
			{\scshape\Large Facultad de Matemática y Computación \par}
			{\scshape\Large MATCOM \par}
				\vspace{3cm}
			{\scshape\Huge Demostración de la condición necesaria y suficiente de las funciones Riemann-Integrables \par}
				\vspace{3cm}
			
				\vfill
			{\Large Autores: \par}
			{\Large Adrián Estévez Álvarez \par}
			{\Large Javier Fontes Basabe \par}
				\vfill
			{\Large Marzo 2025 \par}
	\end{titlepage}


\section{Introducción}
Las integrales son herramientas fundamentales en el Análisis Matemático, especialmente en el cálculo de áreas, volúmenes y otros valores acumulativos. Entre sus distintos enfoques, la integral de Riemann se destaca por su construcción intuitiva basada en sumas de Riemann, lo que permite una aproximación progresiva a la noción de integral a través de particiones del intervalo de integración. \par

Para que una función sea integrable en el sentido de Riemann, debe cumplir ciertas condiciones específicas, como ser acotada. En este contexto, el teorema de necesidad y suficiencia para la integrabilidad de Riemann establece un criterio clave: la existencia de particiones adecuadas y la limitación de la suma de las oscilaciones de la función en subintervalos garantizan su integrabilidad.\par

A lo largo de este documento, se presentará esta condición fundamental, junto con los teoremas y propiedades utilizados en su formulación y demostración. Finalmente, se incluirá un análisis detallado de la demostración del teorema, acompañado de un programa relacionado que ilustra su aplicación.\par

\section{ Teorema:  }
Condición necesaria y suficiente de las funciones Riemann-Integrables
\begin{equation}
    (CNS): \; f \in \mathcal{R}[a, b] \iff \forall \epsilon > 0,  S(f,P)-s(f,P) < \varepsilon. \iff\underline{I} =\overline{I}
\end{equation}

\\

\section{Definiciones y propiedades:}
Para la demostración del teorema se asume que el lector conoce las siguientes definiciones y propiedades:
\\
\item \textbf{Definición 1:} (Definción de Partición.)
\\
Sea $[a, b]$ un intervalo cerrado en $\mathbb{R}$. Una partición $P$ de $[a, b]$ es un conjunto finito de puntos ordenados: 
				\begin{equation}
					\label{def:partición}
					P = \{x_0, x_1, ..., x_n\}: \quad a = x_0 < x_1 < ... < x_n = b 
				\end{equation}
				\begin{equation}
					\label{def:partMax}
					\|P\| = max_{1\leq i \leq n}(x_i - x_{i-1})
\end{equation}

\item \textbf{Definición 2:} (Suma integral de Riemann.)

Sea $P = \{x_0, x_1, \dots, x_n\}$ una partición de $[a,b]$. Definimos la suma integral de $f$ respecto a $P$ como:
\begin{equation}
    \sigma(f, P,\xi_i) = \sum_{i=1}^{n} f(\xi_i) (x_i - x_{i-1}),
\end{equation}
donde $\xi_i$ es un punto en el subintervalo $[x_{i-1}, x_i]$.
\\


\item \textbf{Definición 3:} (Límite de la suma integral de Riemann.)

 Dada una función $f$ y una partición $P$, se denota $\lim \sigma (f,P_k,\{\xi_i\}) = I$, donde $I$ es el límite de la Suma de Riemann asociada a la función $f$ en la partición $P$ con $I \in \mathbb{R}$. Se cumple que: 
				\begin{equation}
					\label{def:limR} 
					\lim \sigma (f,P,\{\xi_i\}) = I \iff \forall \epsilon \quad\exists P_\epsilon \quad P_\epsilon\supset P: \quad |\sigma (f,P_\epsilon,\{\xi_i\}) - I| < \epsilon
				\end{equation}
\textbf{Definición 4:} (Definición de integral de Riemann)
\\
 Dada una función $f$ y un intervalo $[a, b]$, se dice que la función es Riemann-Integrable en el intervalo $[a, b]$, denotado por $f\in\mathcal{R}[a; b]$ si cumple con la siguiente propiedad: 
				\begin{equation}
					\label{def:integR}
					f\in\mathcal{R}[a; b] \iff \exists I:\forall P \quad\lim \sigma (f,P,\{\xi_i\}) = I
				\end{equation}
                \\
                
\item \textbf{Definición 5:} (Sumas de Darboux.)
\\ Dada una función $f$ y una partición $P$ definimos como:
\\
Sumatoria superior de Darboux de $f$ respecto a una partición $P = \{x_0, x_1, \dots, x_n\}$ se define como:
\begin{equation}
    S(f, P) = \sum_{i=1}^{n} M_i (x_i - x_{i-1}),
\end{equation}
donde $M_i = \sup \{ f(x) : x \in [x_{i-1}, x_i] \}$.

La sumatoria inferior de Darboux de $f$ respecto a una partición $P$ se define como:
\begin{equation}
    s(f, P) = \sum_{i=1}^{n} m_i (x_i - x_{i-1}),
\end{equation}
donde $m_i = \inf \{ f(x) : x \in [x_{i-1}, x_i] \}$.

\\
\\

\item \textbf{Definición 6:} (Integral superior e inferior de Darboux.)
\\
Dada una función $f$ y un intervalo $[a, b]$, se denomina Integral Superior e Integral Inferior a los números reales $\overline{\mathrm{I}}$ e $\underline{\mathrm{I}}$, respectivamente, que cumplen la suigientes propiedades:
				\begin{equation}
					\label{def:ingInf}
					\underline{\mathrm{I}} = sup \{s(f, P)\}
				\end{equation}
				\begin{equation}
					\label{def:ingSup}
					\overline{\mathrm{I}} = inf \{S(f, P)\}
				\end{equation}
			\item 




\item \textbf{Propiedad 1:} 
Sea $\sigma(P, \xi_i)$ una suma integral correspondiente a una partición $P$. Entonces, cualquiera que sea la selección de los puntos $\xi_i \in [x_{i-1}, x_i]$ se tiene:
\begin{equation}
    s \leq \sigma(P, \xi_i) \leq S.
\end{equation}

\item \textbf{Propiedad 2:} 
Si $\sigma(P, \xi_i)$ representa una suma integral cualquiera correspondiente a una partición $P$, entonces, se cumple que:

\[
s = \inf_{\xi_i} \sigma(P, \xi_i)   
\]
\[ S = \sup_{\xi_i} \sigma(P, \xi_i) \]


\item \textbf{Propiedad 3:} Siendo $P'$ y  $P $ dos particiones y $P'$ es una partición más fina que $P$, si sus sumas inferiores y superiores son $s'$, $s$, $S'$ y $S$ respectivamente, entonces se cumple :
\begin{equation}
    s \leq s' \leq S' \leq S
\end{equation}

\item \textbf{Propiedad 4:} Sean $P'$ y $P''$  dos particiones en el intervalo $[a,b]$  y $f(x)$ una función acotada en dicho intervalo. Entonces la suma inferior correspondiente a una de las particiones es siempre menor igual que la suma superior correspondiente a la otra partición. 


\item \textbf{Propiedad 5.1:} 
\begin{equation}
   \overline{I} = \inf  S(f, P)\implies \forall \epsilon > 0, \exists \, P_\epsilon \text{ tal que } S(f, P_\epsilon)\leq \  \overline{I} + \epsilon

\end{equation}
\begin{equation}
   \overline{I} = \inf  S(f, P)\implies \forall P \; S(f, P_\epsilon) \geq    \overline{I}

\end{equation}
\item \textbf{Propiedad 5.2:} 
\begin{equation}
   \underline{I} = \sup  s(f, P)\implies \forall \epsilon > 0, \exists \, P_\epsilon \text{ tal que } s(f, P_\epsilon)\geq \  \overline{I} -\epsilon

\end{equation}
\begin{equation}
   \underline{I} = \sup  s(f, P)\implies \forall P \; s(f, P_\epsilon) \leq    \underline{I}

\end{equation}

\\
\item \textbf{Propiedad 6: Propiedad de Acotación de la Integral Superior e Inferior:} Dada una función $f$ y un intervalo $[a, b]$, se cumple que:
				\begin{equation}
					\label{prp:infSup}
					\underline{\mathrm{I}} \leq \overline{\mathrm{I}}
				\end{equation}
		\end{itemize}


\section{Demostración}
\\ Comencemos demostrando que:
\begin{equation}
     f \in \mathcal{R}[a, b] \implies \forall \epsilon > 0,  S(f,P)-s(f,P) < \varepsilon
\end{equation}
De la definición número 4 tenemos:
\begin{align*}
			f\in\mathcal{R}[a; b] &\iff \exists I:\forall P \quad\lim \sigma (f,P,\{\xi_i\}) = I \quad \\
			&\iff \exists I:\forall P \quad\forall \epsilon \quad\exists P_\epsilon \quad P_\epsilon\supset P: \quad |\sigma (f,P_\epsilon,\{\xi_i\} - I| < \epsilon \quad  \\
			&\iff \exists I:\forall \epsilon \quad\forall P \quad\exists P_\epsilon \quad P_\epsilon\supset P: \quad |\sigma (f,P_\epsilon,\{\xi_i\} - I| < \epsilon 
		\end{align*}

Esto significa que:
		\begin{equation}
		    	I - \epsilon < \sigma (f,P,\{\xi_i\} < I + \epsilon
		\end{equation}
		
En específico, tomando $ \sigma (f,P_\epsilon,\{\xi_i\} =S(f,P)$ y $\sigma (f,P_\epsilon,\{\xi_i\}=s(f,P)$ tenemos :
$S(f,P)<\varepsilon+I$
$s(f,P)>I- \varepsilon$

Luego sumando ambas ecuaciones llegamos a:
$S(f,P)-s(f,P)<\varepsilon$

Finalmente tenemos que:
\begin{equation}
     f \in \mathcal{R}[a, b] \implies \forall \epsilon > 0,  S(f,P)-s(f,P) < \varepsilon
\end{equation}
\begin{flushright}
				$\square$
\end{flushright}
            
Demostremos que:
\begin{equation}
\forall \epsilon > 0,\;  S(f,P)-s(f,P) < \varepsilon \implies \underline{I} =\overline{I}
\end{equation}
Supongamos que $\overline{I} \neq \underline{I}$ y $\forall \epsilon > 0,\;  S(f,P)-s(f,P) < \varepsilon$ para encontrar una contradicción.

 Por la propiedad $(6)$ tenemos que $\overline{\mathrm{I}} > \underline{\mathrm{I}}$ , por lo que existe una constante $k$ tal que:
\begin{equation}
   \overline{I} -\underline{I}>k
\end{equation}

Como $\overline{\mathrm{I}}$ y $\underline{\mathrm{I}}$ son infimo y supremo de las sumas de Darboux, tenemos que:
		\begin{align*}
			\forall P:&\\
			&S(f, P) \geq \overline{\mathrm{I}} \\
			&s(f, P) \leq \underline{\mathrm{I}}
		\end{align*}
Restando ambas expresiones:
\begin{equation}
    S(f, P) - s(f, P) \geq \overline{I}- \underline{I}> k.
\end{equation}
Con lo que tenemos:
\begin{equation}
    S(f, P) - s(f, P) > k.
\end{equation}

De nuestra hipótesis tenemos  $\forall \epsilon > 0,\;  S(f,P)-s(f,P) < \varepsilon$. Tomando $\epsilon=k$ llegamos a :
\begin{equation}
    S(f,P)-s(f,P) < k
\end{equation}
Lo cual contradice $(24)$ por lo que llegamos a:
\begin{equation}
\underline{I} \neq\overline{I}\implies \forall \epsilon > 0,\;  S(f,P)-s(f,P) \geq\varepsilon
\end{equation}
Por lo que 
\begin{equation}
    \forall \epsilon > 0,\;  S(f,P)-s(f,P) < \varepsilon \implies \underline{I}  =\overline{I} 
\end{equation}
\begin{flushright}
				$\square$
			\end{flushright}
Procedemos ahora a demostrar  que:
\begin{equation}
\underline{I} =\overline{I}\implies
   f \in \mathcal{R}[a, b] 
\end{equation}
Por definición de integral de Riemann tenemos que:
\begin{equation}
    f\in{R}[a; b] \iff \exists I:\forall P \quad\lim \sigma (f,P,\{\xi_i\}) = I \quad 
\end{equation}
Tenemos:
\begin{equation}
    \overline{I} = \underline{I} = I.
\end{equation}
Luego, dado que  $\overline{\mathrm{I}}$ e $\underline{\mathrm{I}}$ se definen como infimo y supremo de las sumas de Darboux, podemos asegurar que:
			\begin{align*}
				\forall \epsilon > 0  &\quad\exists P_{\epsilon_1}, P_{\epsilon_2}: \\
				s(f, P_{\epsilon_1}) &> I - \epsilon \\
				S(f, P_{\epsilon_2}) &< I + \epsilon
			\end{align*}
Tomando $P = P_{\epsilon_1}\cup P_{\epsilon_2}, $  tenemos que $P\supset P_{\epsilon_1}$ y $P\supset P_{\epsilon_2}$. Por la $Propiedad \; 3 $  llegamos a que 

\begin{equation}
	\forall \epsilon > 0  \quad\exists P: 
\end{equation}
\begin{equation}
    s(f, P) \geq I - \varepsilon
\end{equation}
\begin{equation}
    S(f, P) \leq  I + \varepsilon.
\end{equation}
Sumando estas desigualdades tenemos que  , $\forall P_k, \quad P_k \supset P$::
\begin{equation}
    I - \varepsilon \leq s(f,P_k) \leq S(f, P_k) \leq I + \varepsilon.
\end{equation}
Por el Lema 1 tenemos:
\begin{equation}
  s(f,P_k)  \leq \ \sigma (f,P_k,\{\xi_i\})  \leq S(f, P_k) 
\end{equation}
\begin{equation}
    I - \varepsilon \leq s(f, P_\varepsilon)  \leq \sigma (f,P_k,\{\xi_i\}) \leq S(f, P_\varepsilon) \leq I + \varepsilon.
\end{equation}
Finalmente, se concluye:
\begin{equation}
    I - \varepsilon\leq \sigma(P, \xi_i) \leq  I + \varepsilon.
\end{equation}
De donde :
\begin{equation}
    |\sigma(f, P) - I| < \varepsilon
\end{equation}
 Por lo que podemos afirmar que:
 	\begin{equation}
 	    \exists I:\forall\epsilon >0 \quad\exists P: \forall P_k \supset P \quad|\sigma (f,P_k,\{\xi_i\}) - I| < \epsilon
 	\end{equation}
Con lo que tenemos:
\begin{equation}
\underline{I} =\overline{I}\implies
   f \in \mathcal{R}[a, b] 
\end{equation}

            \begin{flushright}
				$\square$
			\end{flushright}

 Para concluir nuestra demostración tenemos que:
 \begin{equation}
     f \in \mathcal{R}[a, b] \implies \forall \epsilon > 0,  S(f,P)-s(f,P) < \varepsilon
\end{equation}
\begin{equation}
\forall \epsilon > 0,\;  S(f,P)-s(f,P) < \varepsilon \implies \underline{I} =\overline{I}
\end{equation}\begin{equation}
\underline{I} =\overline{I}\implies
   f \in \mathcal{R}[a, b] 
\end{equation}
Aplicando transitividad en estas tres implicaciones llegamos a la conclusion de que son equivalentes. Luego hemos demostrado que:
\begin{equation}
 f \in \mathcal{R}[a, b] \iff \forall \epsilon > 0,\;  S(f,P)-s(f,P) < \varepsilon. \iff\underline{I} =\overline{I}
\end{equation}
 \hfill \(\blacksquare\)

\section{Conclusiones}
En este trabajo se ha demostrado el \textbf{teorema de necesidad y suficiencia para la integrabilidad de Riemann}, el cual establece las condiciones bajo las cuales una función puede ser integrada en este sentido. A través del uso de las \textbf{sumas de Darboux}, se ha visto cómo la existencia de particiones adecuadas permite garantizar la integrabilidad de una función en un intervalo dado. \par

Este resultado es fundamental en el análisis matemático, ya que ayuda a comprender mejor el comportamiento de las funciones y su relación con el concepto de integral. Además, el uso de herramientas computacionales puede facilitar la visualización de estos conceptos y reforzar su comprensión. \par

Se recomienda a los estudiantes demostrar todas las propiedades utilizadas en la demostración del teorema, ya que esto les permitirá afianzar sus conocimientos y comprender mejor los fundamentos de la integral de Riemann. \par

\\\\
Puede ver la visualización de las funciones riemann integrables en el siguiente repositorio: https://github.com/Chikiak/Riemann-Integral


\end{document}
